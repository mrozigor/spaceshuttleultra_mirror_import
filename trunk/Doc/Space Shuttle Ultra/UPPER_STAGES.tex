\documentclass[Space_Shuttle_Ultra_Manual.tex]{subfiles} 
\begin{document}

\section{UPPER STAGES}
\begin{multicols*}{2}
\label{sec:upper-stages}
\renewcommand{\cfttoctitlefont}{\bf}
\localtableofcontents
\noindent
\\
Currently SSU supports the 2 biggest and more powerful Space Shuttle upper stages$\colon$ the Centaur and the Inertial Upper Stage. These 2 upper stages allow the Space Shuttle Ultra add-on to launch payloads to anywhere from GEO to the depths of the Solar System.
\end{multicols*}

\subsection{Centaur}
\begin{multicols*}{2}
\renewcommand{\cfttoctitlefont}{\bf}
\localtableofcontents
\subsubsection{Description}
\noindent
In the 1980s, NASA modified the Centaur upper stage with the intent to use it aboard the Space Shuttle to increase the payload capability of space probes and GEO satellites.
%In the 1980s, NASA modified the Centaur upper stage with the intent to use it aboard the Space Shuttle, thus allowing heavy payloads to be launched to anywhere from GEO to the depths of the Solar System.

\begin{figure}[H]
	\centering
	\captionsetup{justification=centering}
  \includegraphics[width=0.99\hsize]{centaurG.png}
  \caption{Centaur G installed in the payload bay with SSU\_DemoSat}
  \label{fig:centaurG}
\end{figure}

Two versions were developed$\colon$ the Centaur G version was primarily for GEO satellite deployment missions, and the larger, more powerful Centaur G Prime for interplanetary payloads. In the aftermath of the Challenger accident, the Centaur was no longer considered safe enough to be used by the Space Shuttle, and so it was abandoned.
\\
Thrust is provided by 2 RL-10 engines, and the Attitude Control System (ACS) allows 3-axis control of the stage, and also translation in the +Z direction (forward).

\begin{figure}[H]
	\centering
	\captionsetup{justification=centering}
  \includegraphics[width=0.99\hsize]{centaurGP.png}
  \caption{Centaur G Prime installed in the payload bay with SSU\_DemoSat}
  \label{fig:centaurGP}
\end{figure}

\subsubsection{Performance (Centaur G)}
\begin{figure}[H]
	\centering
	\captionsetup{justification=centering}
  \includegraphics[width=0.99\hsize]{Gpayload.png}
  \caption{Centaur G payload capability}
  \label{fig:Gpayload}
\end{figure}
TODO performance

\begin{figure}[H]
	\centering
	\captionsetup{justification=centering}
  \includegraphics[width=0.99\hsize]{Gsize.png}
  \caption{Dimensions of Centaur G and available payload volume}
  \label{fig:Gsize}
\end{figure}
TODO payload envelope

\subsubsection{Performance (Centaur G Prime)}
\begin{figure}[H]
	\centering
	\captionsetup{justification=centering}
  \includegraphics[width=0.99\hsize]{GPpayload.png}
  \caption{Centaur G Prime payload capability}
  \label{fig:GPpayload}
\end{figure}
TODO performance

\begin{figure}[H]
	\centering
	\captionsetup{justification=centering}
  \includegraphics[width=0.99\hsize]{GPsize.png}
  \caption{Dimensions of Centaur G Prime and available payload volume}
  \label{fig:Gpsize}
\end{figure}
TODO payload envelope

\subsubsection{Centaur Integrated Support Structure}
The Centaur Integrated Support Structure (CISS) is the interface between the Centaur stage and the orbiter vehicle. The CISS has a tilt table, to which the Centaur is attached, allowing it to be raised above the payload bay for deployment.
\\
\\
TODO <insert image of empty CISS in PLB?>

\subsubsection{Deployment sequence}
The deployment sequence is similar for both Centaur versions, and is controlled by panel L12.
\\
\\
TODO
\\
\\
Inhibits are placed on the operation of the ACS and of the RL-10 engines, as to protect the orbiter vehicle. At deployment, timers are started to remove those inhibits. The status of those timers is displayed in the HUD, as well as the remaining ACS propellant.

\subsubsection{Autonomous flight control}
After separation from the CISS and the engine inhibits have been removed, the Centaur is controlled by using the standard Orbiter keys. The ''+'' key is used to initiate the start sequence for the RL-10 engines. The start sequence is a 270-second chill-down of the RL-10s concurrent with a propellant settling burn by the forward-thrusting ACS. After the chill-down is complete, RL-10 ignition occurs automatically. After the start sequence is initiated, the time remaining until ignition is shown in the HUD. Currently there are no restrictions on the number of times the RL-10 engines can be started. The ''-'' key is used to shutdown the engines once the desired $\Delta$V has been achieved. During RL-10 burns the attitude in completely controlled by gimballing the engine nozzles.
\\
\\
\WARNING{Engine gimballing is much more powerful than the ACS, so it must be used carefully so the stage is not put into a tumble that might be impossible for the ACS to correct after the burn.}
\\
\\
After all the necessary burns are performed, payload separation is done by pressing the ''Ctrl+J'' key combination.

\subsubsection{Payload Interface}
The connection between the Centaur and its payload is done using a payload adapter. Its exclusive purpose is to interface the payload with the Centaur and is considered a part of the payload, even though on payload deployment the adapter remains with the Centaur.\\
The payload adapter is specified in the Centaur vessel section of the scenario file by using the following 3 entries: ADAPTER\_MESH, defines the path and filename of the mesh file of the adapter; ADAPTER\_OFFSET, defines the offset between the Centaur and the payload in meters (effectively the height of the adapter); ADAPTER\_MASS, defines the mass of the adapter in kilograms.\\
For Centaur payload developers, the payload adapter must be 2.74 meters in diameter at the Centaur end, to correctly interface with the stage. SSU includes an demonstration payload adapter for interfacing the Centaur with SSU\_DemoSat.

\end{multicols*}
\newpage

\subsection{Inertial Upper Stage}
\begin{multicols*}{2}
\renewcommand{\cfttoctitlefont}{\bf}
\localtableofcontents
\subsubsection{Description}
\noindent
The Inertial Upper Stage, or IUS, is a 2-stage solid propellant vehicle used in several Space Shuttle missions to boost satellites into GEO and space probes into Earth escape trajectories.
\\
\begin{figure}[H]
	\centering
	\captionsetup{justification=centering}
  \includegraphics[width=0.99\hsize]{ius.png}
  \caption{Inertial Upper Stage installed in the payload bay with SSU\_DemoSat}
  \label{fig:ius}
\end{figure}

Thrust is provided by one Solid Rocket Motor (SRM) in each stage, and the Reaction Control System (RCS) allows 3-axis control of the stage, and also translation in the +Z direction (forward).\\
The IUS can be installed in the payload bay in 2 possible positions: the forward position or the aft position (for large payloads). The position choice is defined in the mission file (section \ref{sec:mission-files}).

\subsubsection{Performance}
TODO performance
\\
TODO payload envelope

\subsubsection{Airborne Support Equipment}
The Airborne Support Equipment (ASE) is the interface between the IUS and the orbiter vehicle. The ASE has a tilt table, to which the IUS is attached, allowing it to be raised above the payload bay for deployment.
\\
\\
TODO <insert image of empty ASE in PLB?>

\subsubsection{Deployment sequence}
The IUS deployment sequence is controlled by panel L10.
\\
\\
TODO
\\
\\
Inhibits are placed on the operation of the RCS and of the $1^{st}$ stage motor, as to protect the orbiter vehicle. At deployment, timers are started to remove those inhibits. The status of those timers is displayed in the HUD, as well as the remaining RCS propellant.

\subsubsection{Autonomous flight control}
After separation from the ASE and the engine inhibits have been removed, the IUS is controlled by using the standard Orbiter keys. The ''+'' key is used to ignite the SRMs. During SRM burns the attitude in the pitch and yaw axis is controlled by gimballing the engine nozzle, while roll remains under RCS control.
\\
\\
\WARNING{Engine gimballing is much more powerful than the RCS, so it must be used carefully so the stage is not put into a tumble that might be impossible for the RCS to correct after the burn.}
\\
\\
Manual command for the engine gimbal is available and when there is no user input, the rates will be nulled. Once ignited, the SRMs will burn to depletion.\\
After $1^{st}$ stage burnout, its separation is done by pressing the ''Ctrl+G'' key combination. After $1^{st}$ stage separation, the Extendable Exit Cone in the $2^{nd}$ stage will automatically deploy.\\
Due to the nature of the solid propellant motors, fine control of the $\Delta$V is impossible during the burn, so the propellant quantity must be carefully set using the offload capability provided by the appropriate scenario file parameters. The ''LOAD\_STAGE1'' and ''LOAD\_STAGE2'' parameters, followed by a number between 0.5 (50\% offload) and 1.0 (0\% offload), allow to control the initial propellant mass of each SRM. In addition the RCS can be used for velocity fine tuning after the SRM burns.\\
After all the burns are performed, payload separation is done by pressing the ''Ctrl+J'' key combination.

\subsubsection{Payload Interface}
The connection between the IUS and its payload is done using a payload adapter. Its exclusive purpose is to interface the payload with the IUS and is considered a part of the payload, even though on payload deployment the adapter remains with the IUS.\\
The payload adapter is specified in the IUS vessel section of the scenario file by using the following 3 entries: ADAPTER\_MESH, defines the path and filename of the mesh file of the adapter; ADAPTER\_OFFSET, defines the offset between the IUS and the payload in meters (effectively the height of the adapter); ADAPTER\_MASS, defines the mass of the adapter in kilograms.\\
For IUS payload developers, the payload adapter must be 2.89 meters in diameter at the IUS end, to correctly interface with the stage. SSU includes an demonstration payload adapter for interfacing the Centaur with SSU\_DemoSat.

\end{multicols*}
\end{document}