\documentclass[13pt, letter,final]{article}
\special{papersize=8.5in, 11in}
\usepackage[margin=1in]{geometry}
\usepackage{fancyhdr}
\pagestyle{fancy}
\lhead{}
\chead{}
\rhead{Space Shuttle Ultra Manual\\ Rev. B}
\lfoot{}
\cfoot{\thepage}
\rfoot{\leftmark\\ \rightmark}
\renewcommand{\footrulewidth}{0.4pt}
\usepackage[T1]{fontenc}
\usepackage[scaled]{helvet}
\renewcommand*\familydefault{\sfdefault}
\usepackage[pdftex]{graphicx}
\usepackage{hyperref}
\usepackage{multicol}
\usepackage{stfloats}
\usepackage{float} 
\usepackage{subfiles}
\usepackage{grffile}
\sffamily
\graphicspath{{Images//}}

\begin{document}
\pagenumbering{roman}
\begin{titlepage}

\begin{center}

\includegraphics[width=0.35\textwidth]{SSP.jpg}\\[1cm]    

\textsc{\LARGE Space Shuttle Ultra}\\[1.5cm]

\textsc{\Large Version 1.XX  Rev. A}\\[0.5cm]

\huge \bfseries SSU Operations Manual\\[0.4cm]

\vfill

{\large \today}

\end{center}

\end{titlepage}
\begin{multicols}{2}
\section*{\large PREFACE}
\addcontentsline{toc}{section}{PREFACE}
Space Shuttle Ultra (SSU) is an addon for Orbiter Space Flight Simulator.  The purpose of this addon is to fully simulate the NASA's Space Transportation System Program.  Currently only a few elements have been completed and work on others is ongoing.\\
\\
The basis for this addon was the Space Shuttle Deluxe but through adding additional subsystems and taking full advantage of the 2010 version of Orbiter, the current SSU has few similarities to the original Deluxe.  Currently, SSU simulates a number of systems, displays, and procedures of the real shuttle and can be used along with real NASA Flight Data File (FDF) checklists to complete tasks.  These checklists can be found at \url{http://www.nasa.gov/centers/johnson/news/flightdatafiles/index.html}, and provide a good reference for other procedures.\\
\\
Other good NASA references are the Shuttle Crew Operations Manual (SCOM), the DPS Dictionary, and the various Workbooks and Handbooks that are available on the Internet (these can also be found at the above link).  While reading these are not required at this time (all pertinent information is provided in SSU documents) eventually when systems are fully simulated, we will point users to these documents for information deeper than a basic summery of the system.\\
\\
This document contains the condensed material taken from various NASA documents as well and Orbiter and SSU specific information.  The goal of this document is to provide a typical Orbiter User the information need to perform basic SSU flights as well as  to aid in basic custum mission creation.  Separate documents will be provided for developers who would like to create a SSU compatible payloads and scenarios.\\
\\
This document is formated to look the same as the SCOM to facilitate changing from one document to the other.  Additional information or clarification is presented in three formats: notes, cautions, and warnings. Notes provide amplifying information of a general nature. Cautions provide information and instructions necessary to prevent hardware damage or malfunction (not yet simulated). Warnings provide information and instructions necessary to ensure crew safety (also not simulated). The formats in which this material appears are illustrated below.\\
\\
\begin{tabular}{c}
	\bfseries NOTE\\[0.3cm] A barberpole APU/HYD READY TO\\ START talkback will not inhibit a start.
\end{tabular}
\\[0.5cm]
\begin{tabular}{|c|}
	\hline
	\\[0.1cm]
	\bfseries CAUTION\\[0.3cm] After an APU auto shutdown, the APU\\ FUEL TK VLV switch must be taken to\\ CLOSE prior to inhibiting auto shutdown\\ logic. Failure to do so can allow the fuel\\ tank isolation valves to reopen and flow\\ fuel to an APU gas generator bed that is\\ above the temperature limits for safe\\ restart.\\
	\hline
\end{tabular}
\\[0.5cm]
\begin{tabular}{||c||}
	\hline \hline\\
	\bfseries WARNING\\[0.3cm] The FUEL CELL REAC switches on panel\\ R1 are in a vertical column with FUEL\\ CELL 1 REAC on top, FUEL CELL 3 REAC\\ in the middle, and FUEL CELL2 REAC on\\ the bottom. This was done to allow the\\ schematic to be placed on the panel.\\ Because the switches are not in numerical\\ order, it is possible to inadvertently close\\ the wrong fuel cell reactant valve when\\ shutting down a fuel cell.\\
	\hline \hline
\end{tabular}
\\
\end{multicols}

\newpage
\tableofcontents
\newpage

\pagenumbering{arabic}
\begin{multicols}{2}
\section{\large GENERAL DESCRIPTION}

\begin{tabular}{|p{6.9cm}  p{0.25cm}|}
	\hline
	&\\[0.1cm]
	CONTENTS & \\[0.4cm]
	1.1	OVERVIEW & 2\\
	1.2	ORBITER AND SSU & 7\\
	1.3	COMPONENTS OVERVIEW & \\
	\hline
\end{tabular}
\\[0.4cm]
The section provides general background information about the obiter, its configuration and coordinate system, the nominal mission profile, and general procedures followed during a shuttle mission. It also briefly discusses components, such as the external tank and solid rocket, that are not included in the next section on orbiter systems.\\
\\
Also included in this section is keyboard commands for SSU but will not include standard Orbiter keyboard commands. See Orbiter.pdf for standard Orbiter keyboard commands.
\newpage

\subsection{\large OVERVIEW}

\begin{tabular}{|p{6.9cm} p{0.25cm}|}
	\hline
	&\\[0.1cm]
	CONTENTS & \\[0.4cm]
	Space Shuttle Overview & 2\\
	Nominal Mission Profile & 2\\
	Launch and Landing Sites & 4 \\
	Shuttle Location Codes &  4\\
	\hline
\end{tabular}

\end{multicols}
\includegraphics[width=1\textwidth]{Space Shuttle Stats.jpg}
\begin{multicols}{2}

\subsection*{\large Space Shuttle Overview}
\addcontentsline{toc}{subsection}{Space Shuttle Overview}
The space shuttle system consists of four primary elements: an orbiter spacecraft, two solid rocket boosters (SRBs), and external tank (ET) to house fuel and oxidizer, and three space shuttle main engines (SSMEs). The shuttle can transport payloads into near Earth orbit 100 to 312 nm (185 to 577 km) above the Earth. Payloads are carried in a bay 15 feet in diameter and 60 feet long. Major system requirements are that the orbiter and the two SRBs be reusable.\\
\\
The orbiter has carries a flight crew of up to eight persons. The nominal mission is 4 to 16 days in space. The crew compartment has a shirtsleeve environment, and the acceleration load is never greater than 3 g's. In its return to Earth, the orbiter has a crossrange maneuvering capability of about 1,100 nm.\\

\subsection*{\large Nominal Mission Profile}
\addcontentsline{toc}{subsection}{Nominal Mission Profile}
SSU has reached a point of development that almost a full mission profile can be simulated. Continued development of Entry, from Entry Interface (EI) to terminal area energy management (TAEM), will see implementation of the entry guidance and displays that will aid in executing reentry but for now it must be done manually with limited instrumentation.\\
\\
\begin{center}
\subsection*{Launch}
\end{center}
In launch configuration, the orbiter and two SRBs are attached to the ET in a vertical (nose-up) position on the launch pad.\\
\\
The three SSMEs, fed liquid hydrogen fuel and liquid oxygen oxidizer from the ET, are ignited first $\sim$T-6 seconds. Once proper operation and thrust level has been varified, a signal is sent at T-0 seconds to the SRBs which ignite and the stack is released from the launch pad.\\
\\
After approximately 2 minutes into the ascent phase, the two SRBs have consumed their propellant and are jettisoned from the ET. This is triggered by a separation signal from the orbiter.\\
\\
The orbiter and ET continue on to orbit while the SRBs begin to descend back to Earth. At a predetermined altitude, parachutes are deployed and the boosters spashdown in the ocean.
The boosters are recovered and reused.\\
\\
After approximately 8 and a half minutes after launch, the three main engines undergo main engine cutoff (or MECO), and the ET is jettisoned on command from the orbiter.\\
\\
The forward and aft reaction control system (RCS) jets provide attitude control, translate the orbiter away from the ET at separation, and maneuver the orbiter to burn attitude prior to the orbital maneuvering system (OMS) burn. The ET continues on a ballistic trajectory and enters the atmosphere, where it disintegrates.\\
\\
\begin{center}
\subsection*{Orbit Insertion and Circularization}
\end{center}
The nominal ascent profile, referred to as ``direct insertion," places the vehicle in a temporary elliptical orbit at MECO. Orbital altitudes can very depending on mission requirements. The crew performs an OMS burn, designated as ``OMS 2'', to stabilize the orbit. This burn can add anywhere between 200 to 550 fps to the vehicle's orbital velocity, as necessary.\\
\\
In cases of severe performance problems during the ascent, the vehicle may find itself well short of the expected MECO velocity, and even suborbital. In such cases, the crew performs what is call an ``OMS 1'' burn, which raises the orbit to a safe altitude. They then perform an OMS 2 burn to stabilize that orbit.\\
\\
When simulating early missions with SSU. The orbiter will perform what was known as a ``standard insertion''. This will place the orbiter in a heads down suborbital orbit at MECO and will require an OMS 1 burn.  This ascent profile was used for the first ten missions, STS-1 though STS-41B.\\
\\
\begin{center}
\subsection*{Orbit}
\end{center}
On orbit, the forward and aft RCS jets provide attitude control of the orbiter, as well as any minor translation maneuvers along a given axis. The OMS engines are used to perform orbital transfers, such as those done to rendezvous with the International Space Station (ISS). Mission objectives while in orbit has ranged from ISS assembly and logistics, payload deployment and retrieval, to scientific experiments.  Also several planed, but not flown, missions are planed to be simulated with SSU including shuttle-Centaur flights and posible Vandenberg missions.
\\
\begin{center}
\subsection*{Deorbit}
\end{center}
At the completion of orbital operations, the RCS is used to orient the orbiter in a tail-first attitude. The two OMS engines are burned to lower the orbit such that the vehicle enters the atmosphere at a specific altitude and range from the landing site. The deorbit burn usually decreases the vehicle's orbital velocity anywhere from 200 to 550 fps, depending on orbital altitude.  When the deorbit burn is complete, the RCS is used to rotate the orbiter's nose forward for entry. The RCS jets are used for attitude control until atmospheric density is sufficient for the pitch, roll, and yaw aerodynamic control surfaces to become effective.
\\
\begin{center}
\subsection*{Entry}
\end{center}
Orbiter reentry is normally controlled automatically by the Digital Autopilot (DAP) from entry interface (EI) through TAEM, to $\sim$ mach 2, where the CDR takes control of the orbiter. This autopilot has yet to be simulated in SSU but manual entry is possible with third party addons that provide more situational awareness then what is currently provided.  A detailed description of reentry procedures will be included later in this document.
\\
\begin{center}
\subsection*{TAEM}
\end{center}
TAEM (terminal area energy management) guidance steers the orbiter to one of two heading alignment cones (HAC), which are located to and on either side of the runway centerline on the approch end.  Again, the guidance that would aid the CDR or PLT during this phase of the flight has yet to be implemented but is still posible without it. Detailed desriptions will be included later in this document.
\\
\begin{center}
\subsection*{Landing}
\end{center}
Approach and landing easily performed without TAEM guidance. After runway acqusition on rollout from the HAC one can visually fly the orbiter to runway. Detailed desriptions of final approach and landing will be included later in this document.\\

\subsection*{\large Launch and Landing sites}
\addcontentsline{toc}{subsection}{Launch and Landing sites}
During the Shuttle program, The Kennedy Space Center (KSC) in Florida was used for all shuttle launches. Currently in SSU that is also true. Eventually, SSU hopes to simulate West Coast shuttle operations based at Vandenberg Air Force Base as it was originaly planned until the Challenger accident in 1986. Shuttle landings occur at KSC, also, as well as at Edwards Air Force Base in California. Contingency landing sites are also provided in the event the orbiter must return to Earth in an emergency.\\
\\
\begin{tabular}{c}
	\bfseries NOTE\\[0.3cm] KSC is currently the only base that is\\ included with SSU.  Eventually Edwards,\\ White Sands, VAFB, and the abort\\ landing sites will be included.
\end{tabular}
\\[0.5cm]
A 035$^{\circ}$ azimuth launch places the spacecraft in an orbital inclination of 57$^{\circ}$, which means the spacecraft in its orbital trajectories around Earth will never exceed an an Earth latitude higher or lower than 57$^{\circ}$ north or south of the equator. A launch path from KSC at an azimuth of 090$^{\circ}$ (due east from KSC) will place the spacecraft in an orbital inclination of 28.5$^{\circ}$.
\\
These two azimuths, 035$^{\circ}$ and 090$^{\circ}$, represent the current launch limits from KSC. Any azimuth angles further north or south would launch the spacecraft over a habitable land mass, adversely affect safety provisions for abort or vehicle separation conditions, or present the undesirable possibility that the SRB or external tank could land on inhabited territory.
\\
\subsection*{\large Shuttle Location Codes}
\addcontentsline{toc}{subsection}{Shuttle Location Codes}
Orbiter location codes enable crewmembers to locate displays and controls, stowage comparments and lockers, access panels, and wall-mounted equipment in the orbiter crew compartments. The crew compartments are the flight deck, middeck, and airlock. Because of compartment functions and geometry, each has a unique location coding format.
\\
Currently SSU only simulates the flight deck to the degree that location coding must be followed. Soon panels in the middeck will be simulated and at that time the middeck location coding will be included in this manual.
\\
\begin{center}
\subsection*{Flight Deck Location Codes}
\end{center}
A flight deck location code consists of two or three alphanumeric characters. The first character is the first letter of a flight dect surface as addressed while sitting in the commander/pilot seats. The characters are:
\\
\begin{tabular}{r c l}
	L& - &Left\\
	R& - &Right\\
	F &-& Forward\\
	A &-& Aft\\
	C &- &Center Console\\
	O &-& Overhead\\
	S &-& Seats\\
	W &-& Window\\
\end{tabular}
\\
\end{multicols}
\includegraphics[width=0.85\textwidth]{Crew Cabin (Cutaway).jpg}
\begin{multicols}{2}
\\
\includegraphics[width=0.45\textwidth]{FD_Codes_Chart.jpg}
\\
The second and third characters are numerics identifying the relative location of componets on each flight deck surface. The numbering system philosophy is summerized in the table at left.
\\
\end{multicols}
\includegraphics[width=1\textwidth]{Flight_Deck_Loc_Codes_1.jpg}

\\


\includegraphics[width=1\textwidth]{Flight_Deck_Loc_Codes_2.jpg}

\vfill

\newpage
\subsection{\large ORBITER AND SSU}

\begin{tabular}{|p{7cm} p{0.25cm}|}
	\hline
	&\\[0.1cm]
	CONTENTS & \\[0.4cm]
	SSU Keyboard Commands & 7\\
	Camera Views & 8\\
	Payload Operations Overview & 9\\
	\hline
\end{tabular}


\subsection*{\large SSU Keyboard Commands}
\addcontentsline{toc}{subsection}{SSU Keyboard Commands}
The ultimate goal of SSU is to provide a complete simulation of the Space Shuttle.  This means that most of the imput is done with in-simulation controls (ie. cockpit switches, GNC keyboards, and dialog windows). This results in very few keyboard commands to operate the shuttle.

\subsection*{General}
Ctrl+A - toggle between controlling RCS thrusters and RMS motion\\
Ctrl+G - arm landing gear\\
G - deploy landing gear\\
Ctrl+B - toggle fully open/close speedbrake\\
comma - open speedbrake by 5 percent\\
period - close speedbrake by 5 percent\\
B - turn off launch autopilot\\
C - toggle automatic throttling\\
Ctrl+L - toggle PLB lights on/off

\subsection*{Alternate Translation Commands (valid only in RCS is in Rot mode)}
Left/Right Arrow - left/right translation (equivalent to 1/3 on Numpad)\\
Up/Down Arrow - up/down translation (equivalent to 8/2 on Numpad)\\
Insert/Delete - forward/aft translation (equivalent to 9/6 on Numpad)\\

\subsection*{RMS}
Ctrl+Enter - grapple\\
Ctrl+Backspace - release\\
Ctrl+O - toggle between Coarse and and Vern rates\\
\newpage

\subsection*{\large Camera Views}
\addcontentsline{toc}{subsection}{Camera Views}
SSU includes the four payload bay cameras and the docking port centerline camera. The PLB cameras are controled via their switches on panel A7U on the flight deck and are no longer control with a dialog window.

\subsection*{Navigating the Virtual Cockpit}
Changing between Virtual Cockpit (VC) views is identical to the system used in the default Atlantis but with several more positions around the cockpit that we call stations. You can switch between different stations using the Ctrl+Arrow key combination (See Chart below for all combinations.) The Commander (CDR) Station is the front left seat on the flight deck, while the Pilot (PLT) station is the right seat (while looking forward).\\

The table below is set up to show the different ways to move about the crew module. The first column is the camera position you are in and the other columns show the views you can change to using the Ctrl+Arrow key combination at the top of the table. For additional assistance in navigating the views, the name of the view is shown for a few seconds at the top of the screen during the simulation. The names in the table are identical to those that appear on-screen.\\

\begin{tabular}{|l|l|l|l|l|}
	\hline
	Cockpit View & Left & Right & Up & Down \\
	\hline \hline
	CDR & Port Workstation & PLT & ODS & MS1 \\
	\hline
	PLT & CDR & Stbd Workstation & ODS & MS2 \\
	\hline
	MS1 & Port Workstation & MS2 & CDR & ODS \\
	\hline
	MS2 & MS1 & Stbd Workstation & PLT & ODS\\
	\hline
	Port Workstation & RMS Station & CDR & ODS & Middeck\\
	\hline
	Stbd Workstation & PLT & Aft Pilot & ODS & Aft Workstation\\
	\hline
	Aft Workstation & Stbd Workstation & Port Workstation & RMS Station & MS1\\
	\hline
	Aft Pilot & stbd Workstation & RMS Station & ODS & Aft Workstation\\
	\hline
	RMS Station & Aft Pilot & Port Station & ODS & Aft Workstation\\
	\hline
	RMS EE & RMS Elbow & - & - & RMS Station\\
	\hline
	RMS Elbow & - & RMS EE & - & RMS Station\\
	\hline
	PLB Camera A & PLB Camera D & PLB Camera B & RMS EE & ODS\\
	\hline
	PLB Camera D & PLB Camera C & PLB Camera A & RMS EE & ODS\\
	\hline
	PLB Camera B & PLB Camera A & PLB Camera C & RMS EE & ODS\\
	\hline
	PLB Camera C & PLB Camera B & PLB Camera D & RMS EE & ODS\\
	\hline
	ODS & - & - & PLB Camera D & Aft Pilot\\
	\hline
\end{tabular}
\newpage
\begin{multicols}{2}
\subsection*{\large Payload Operations Overview}
\addcontentsline{toc}{subsection}{ Payload Operations Overview}


\newpage
\subsection{\large Components Overview}

\begin{tabular}{|p{7cm} p{0.25cm}|}
	\hline
	&\\[0.1cm]
	CONTENTS & \\[0.4cm]
	Orbiter & \\
	External Tank & \\
	Solid Rocket Boosters & \\
	\hline
\end{tabular}
\subsection*{\large Orbiter}
\addcontentsline{toc}{subsection}{ Orbiter}

The Orbiter is the manned vehicle component of the Space Shuttle System and is the only component to reach orbital velocities.  The orbiter itself is divided into nine major sections: (1) forward fuselage, which includes the crew compartment, (2) wings, (3) midfuselage, (4) payload bay doors, (5) aft fuselage, (6) forward RCS, (7) vertical tail, (8) OMS/RCS pods, (9) body flap.  Many of these sections are only structural sections of the orbiter and are thus not simulated by SSU. The following descriptions discuss the key sections as they pertain to the user in simulation and point out important systems contained in them.

\begin{center}
\subsection*{Forward Fuselage}
\end{center}
The forward fuselage contains the crew compartment, forward RCS module, nose cap, nose gear wheel well with associated gear and doors.  The forward fuselage also contains various antennas, deployable air data probes and the openings for the two star trackers.
\\
\begin{center}
\subsection*{Crew Compartment}
\end{center}
The crew compartment arrangement consists of three levels, the flight deck, the middeck and the lower equipment bay. Currently only the flight deck and the middeck are simulated in any capacity. The compartment has a side hatch for normal ingress and egress, a hatch into the airlock from the middeck, and a hatch from the airlock into the payload bay for extravehicular activity and payload bay access (Currently none of these hatches are simulated).

\end{multicols}
\includegraphics[width=1\textwidth]{Orbiter.jpg}
\begin{multicols}{2}


\end{multicols}

\newpage
\subfile{SSU_SYSTEMS}
\newpage
\subfile{FLIGHT_DATA_FILES}
\newpage
\subfile{MISSION_FILES}
\newpage
\subfile{SCENARIO_FILES}
\end{document}