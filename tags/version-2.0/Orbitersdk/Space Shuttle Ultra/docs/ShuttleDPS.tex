\documentclass[a4paper, twocolumn, 11pt]{book}
\usepackage[ansinew]{inputenc}
\usepackage[english]{babel}
\usepackage[T1]{fontenc}
\usepackage{graphicx}
\usepackage{geometry}
\usepackage{amsmath}
\usepackage[pdftex]{hyperref}
\usepackage{makeidx}
\makeindex
\begin{document}
\title{The Space Shuttle Data Processing System}
\author{Dennis Krenz}
\maketitle
\tableofcontents
\chapter{Introduction}
\section{What is this about?}
This document is meant as compilation of information from various sources
regarding the Data Processing System of the Space Shuttle for the Project
``Space Shuttle Ultra''.
\chapter{The Shuttle Bus}
As precedessor of the MIL-STD-1553, it is a serial bus system, using two twisted
lines per bus. Hardware is coupled to the bus over optical couplings. The
maximum data rate is 1 MBit/s. Communication is strictly hierachical, with a
single Bus Control Element (BCE) controlling the bus and polling the connected
hardware.
\section{The Multiplexer Interface Assembly}
The multiplexer interface assembly (MIA) is the connection between hardware and
\chapter{The GPCs}
\section{The origins}
The AP-101\index{AP-101} GPC family is derived from the radiation-hardened IBM
System/4Pi series, which in turn
is based on the famous IBM System/360 series. The AP-101 is a 32 bit single CPU
computer, with 16 general-purpose registers and 8 floating point registers. The
CPU is only capable of addressing 64 KBytes memory directly, but can adress more
memory by using paging.
\section{Structure of the GPC}
The GPC consists of CPU, memory and a special Input/Output Processor (IOP)
\index{IOP@Input/Output Processor}, which is a simple processing unit capable of
managing 24 serial Shuttle Bus systems and many discrete lines. The IOP can
access the Shuttle memory directly, freeing capacity of the AP-101 CPU for the
actual flight software.
\section{The different flown versions of the GPC}
\subsection{AP-101A}
\index{AP-101!AP-101A}
Initial version of the AP-101, used in the B-1B and Enterprise.
\subsection{AP-101B}
The AP-101B\index{AP-101!AP-101B} is the first version of the GPC, which reached
space.
\subsection{AP-101F}
The AP-101F\index{AP-101!AP-101F} update replaced the old magnetic core memory
of the early GPCs with dynamic random access memory. Because this memory looses
its contents when powered off, the new DRAM memory gets kept alive by a buffer battery.

\subsection{AP-101S}
The current version AP-101S\index{AP-101!AP-101S} is a small improvement of the
AP-101F, which reduced the size of the hardware so much, that it was possible to
fit both CPU and IOP into a single LRU.
\chapter{The Multiplexers/Demultiplexers}
The multiplexers/demultiplexers(MDMs)\index{MDM@Multiplexer/Demultiplexer} take data from various input sources (serial,
discrete lines, analog signals) and format them for the Shuttle Bus
(multiplexing) or take data
received from the Shuttle Bus and distribute it to the connected hardware
(demultiplexing)
Each MDM has a small program memory (PROM), which contains lists of I/O actions
which get triggered by the polling GPC. Each can be thought as a format, which
tells the MDM to which hardware the GPC wants to output the following data
words.

\begin{appendix}
\chapter{Glossary}

\end{appendix}
\printindex
\end{document}