\documentclass[Space_Shuttle_Ultra_Manual.tex]{subfiles} 
\begin{document}
\begin{multicols}{2}

\section{\large SSU SYSTEMS}

\begin{tabular}{| p{0.3cm}  p{6.7cm}  p{0.25cm}|}
	\hline
	 & & \\[0.1cm]
	Contents & & \\[0.4cm]
	2.1 & AUXILIARY POWER UNIT/ HYDRALICS & \\
	2.2 & DATA PROCESSING SYSTEM & \\
	2.3 & DEDICATED DISPLAY SYSTEMS & \\
	2.4 & GUIDANCE NAVIGATION AND CONTROL & \\
	2.5 & MAIN PROPULSION SYSTEM & \\
	2.6 & MECHANICAL SYSTEMS & \\
	2.7 & ORBITAL MANEUVERING SYSTEM & \\
	2.8 & ORBITER DOCKING SYSTEM & \\
	2.9 & PAYLOAD DEPLOYMENT AND RETIEVAL SYSTEM & \\
	2.10 & REACTION CONTROL SYSTEM & \\
	\hline
\end{tabular}
\\[0.4cm]
This section discusses in greater detail each of the orbiter systems that are currently simulated. The subsections are organized alphabetically, with detailed internal table of contents provided for each.\\
\\
These are not ment to provide all information about the real orbiter system but to provide a working knowlege required to understand what is happening in the simulation. For full, detailed reading, each subsection provides references to relavent sections in the Shuttle Crew Operations Manual (SCOM) and should be read for a better understanding for that system.  Thoses who are just after learning the basics will be satisfied by this document.

\newpage
\subsection{\large AUXILIARY POWER UNIT/HYDRAULICS (APU/HYD)}

\begin{tabular}{|p{6.9cm} p{0.25cm}|}
	\hline
	&\\[0.1cm]
	CONTENTS & \\[0.4cm]
	Description &\\
	System Overview & \\
	Operations & \\
	Relavent Panels and Displays &  \\
	\hline
\end{tabular}
\\[0.4cm]
\subsection*{Description}
The orbiter has three independant hydraulic systems. Each system provides hydraulic pressure to position hydraulic actuators for:
	

\end{multicols}
\end{document}
