\documentclass[Space_Shuttle_Ultra_Manual.tex]{subfiles} 
\begin{document}

\section{SSU SYSTEMS}
\begin{multicols*}{2}
\renewcommand{\cfttoctitlefont}{\bf}
\localtableofcontents
\noindent
This section discusses in greater detail each of the orbiter systems that are currently simulated. The subsections are organized alphabetically, with detailed internal table of contents provided for each.\\
\\
These are not ment to provide all information about the real orbiter system but to provide a working knowlege required to understand what is happening in the simulation. For full, detailed reading, each subsection provides references to relavent sections in the Shuttle Crew Operations Manual (SCOM) and should be read for a better understanding for that system.
\end{multicols*}

\subsection{AUXILIARY POWER UNIT/HYDRAULICS (APU/HYD)}
\begin{multicols*}{2}
\renewcommand{\cfttoctitlefont}{\bf}
\localtableofcontents
\subsubsection{Description}
The orbiter has three independant hydraulic systems. Each system provides hydraulic pressure during launch and entry to control the SSMEs, the Orbiter's aerosurfaces, and other systems.
The APUs are started 5 minutes before launch and shut down shortly after MECO.
The day before entry, a single APU is started for the FCS checkout.
A single APU is started 5 minutes before the deorbit burn (this is done to ensure at least 1 APU is functioning before committing to entry). The remaining APUs are started 13 minutes before Entry Interface.
All 3 APUs are shut down after landing.
\end{multicols*}

\subsection{Data Processing System (DPS)}
\begin{multicols*}{2}
\renewcommand{\cfttoctitlefont}{\bf}
\localtableofcontents
The DPS consists of the shuttle's General Purpose Computers (GPCs), associated systems, and the software run by the GPCs. The 11 MDUs (Multifunction Display Units) are also part of the DPS.
\subsubsection{GPCs}
The real shuttle has 5 identical computers. Up to 4 of the 5 GPCs run the Primary Avionics Software System (PASS). The remaining computer runs the Backup Flight System (BFS). The PASS software is further divided into 3 Major Functions: \textit{GNC} (Guidance, Navigation \& Control), \textit{SM} (Systems Management) and \textit{PL} (Payload) software. The \textit{GNC} software is responsible for controlling the orbiter during flight. During critical phases of flight, such as launch and entry, multiple GPCs will run the PASS \textit{GNC} software simultaneously; this provides redundacy if one of the GPCs fails. The \textit{SM} software monitors various orbiter systems. The \textit{PL} software is not used during flight. The BFS was written separately from the PASS, and implements a subset of the PASS \textit{GNC} functions. The BFS is meant to be used in the event of a PASS failure. \\
\\
The \textit{GNC} major function is divided into multiple OPS. Each OPS represents a different phase of flight. OPS 1 is used for launch, OPS 2 is used on-orbit, and OPS 3 is used for deorbit and entry. The GPC only has enough memory to store one OPS at a time, so the PASS software is divided into multiple memory configurations. Each memory configuration contains one OPS (except for MC 1, which is used during launch, and contains both OPS 1 (launch) and OPS 6 (RTLS)). To change from one OPS to another, the appropriate memory configuration has to be loaded onto the GPCs.
Each OPS is further divided into Major Modes, which relate to specific phases of the mission. For example, OPS 2 (on-orbit) has 2 Major Modes: MM 201 (orbit coast) and MM 202 (Mnvr Exec). MM 202 is used for performing OMS burn, while MM 201 is used otherwise.
\\
At the moment, SSU only simulates the PASS \textit{GNC} software. Also, loading different memory configurations into the GPCs is not simulated. SSU assumes only one GPC is running, and does not simulate multiple GPCs performing the same operations as part of a redundant set.
\subsubsection{Multifunction Display Units (MDUs)}
The shuttle originally had 4 CRT displays, and multiple analog instruments.
The CRTs allowed the crew to interact with the shuttle computers, while the analog instruments displayed susbsystem status and flight instruments.
Starting with STS-101, the analog instruments were replaced with the MDUs.
The shuttle has 11 MDUs: CDR 1 and 2 on panel F6; CRT 1, 2, and 3; MFD 1 and 2 on panel F7; PLT 1 and 2 on panel F8; CRT 4 on panel R12; and AFD 1 in the aft station.
In real life, the MDUs display either DPS displays, flight instrument displays, or subsystem status displays.
The flight instruments and subsystem status displays replace the analog instruments, while the DPS displays are almost identical to the CRT displays.\\
\\
In SSU, each MDU is an Orbitersim MFD. CRT MFD, which is part of SSU, simulates the shuttle MEDS displays. In the future, SSU will only display accurate displays in the MDU; some displays have not been implemented yet, and so Orbitersim MFD equivalents have to be used.
Section \ref{sec:dps-displays} describes the DPS displays that have been implemented so far.
The 3 subsystem status displays (\textit{OMS/MPS}, \textit{APU/HYD} and \textit{SPI}) have been implemented in CRT MFD.
The flight instrument displays are only partially implemented. All displays in the Ascent/Entry Primary Flight Display are working except for the ADI and HSI.
\end{multicols*}

\subsection{DPS Displays}
\begin{multicols*}{2}
\renewcommand{\cfttoctitlefont}{\bf}
\localtableofcontents
\label{sec:dps-displays}
The NASA DPS Dictionary describes each display in detail. This section lists the displays that have been implemented so far and describes the differences between the real shuttle and the SSU implementation.

\subsubsection{ASCENT TRAJ}
This display is used in MM 102 and MM 103. The ITEMs on this display are related to abort options and are not supported by SSU.

\subsubsection{UNIV PTG}
This display is used in MM 201, and is used to control the attitude of the orbiter. Most of the functions in this display have been implemented. ITEM 8 (TGT ID) only supports an entry of 2 at the moment, and ITEMS 9-13 are not supported. ITEM 20 (TRK) is not supported. Finally, ITEMs 22-24 (which affect how the attitude error is displayed) are not implemented.

\subsubsection{OMS MNVR EXEC}
This display is used in MM 104 (OMS 1 MNVR EXEC), MM 105 (OMS 2 MNVR EXEC), MM 106 (OMS 2 MNVR COAST), MM 202 (ORBIT MNVR EXEC), MM 301 (DEORB MNVR COAST), MM 302 (DEORB MNVR EXEC) and MM303 (DEORB MNVR COAST). It is used mainly to perform OMS enginer burns to change the shuttle's orbit.
This display is almost completely implemented in SSU. ITEMs 28-40 (OMS gimbal check, FWD RCS dump and SURF DRIVE) have not been implemented yet.

\subsubsection{DAP CONFIG}
This display is used in MM 201 and MM 202, and control the Digital Autopilot (DAP) settings. In real life, there are 15 DAP A configurations and 15 DAP B configurations; at any time, 1 DAP A and 1 DAP B configuration is active, and the crew selects between DAP A and B using the PBIs on Panels C3 and A6. In SSU, there is only 1 DAP A configuration and 1 DAP B configuration. As a result, ITEMS 1 and 2 (which select the active DAP A \& B configuration) are not implemented. Also ITEMs 3 and 4 (which, in real life, select a DAP configuration and load it into the EDIT column) simply select between loading DAP A and DAP B into the EDIT column.

\subsubsection{ORBIT TGT}
This display is used in MM 201 and MM 202 to compute rendezvous burns. In real life, the state vectors for the rendezvous target are uploaded from Mission Control. In SSU, the name of the target vessel is specified in the scenario file.
% TODO: reference scenario file
The real-life ORBIT TGT display can load rendezvous targets by specifying a TGT NO (ITEM 1); in SSU, each parameter has to be set individually. SSU doesn't support the EL parameter (ITEM 6), which allows the burn TIG to be computed to match a desired elevation angle; instead, the TIG must be specified.
SSU can only be used to compute the T1 burn (ITEM 28), and not the T2 burn. In real life, the T2 burn computations are not used.

\subsubsection{ENTRY TRAJ}
These displays are used during entry to monitor the shuttle's trajectory.

\subsubsection{VERT TRAJ}
These displays are used during TAEM to monitor the shuttle's trajectory.

\subsubsection{HORIZ SIT}
This display is used during deorbit and entry to specify the landing site and monitor the position of the shuttle relative to the HAC and the runway.
The HORIZ SIT display in SSU is simplified compared to the real life version. Only ITEMs 3, 4, 6 and 41 are supported. These ITEM 41 selects the landing site, ITEMs 3 and 4 switch between the primary and secondary runway, and ITEM 6 switches between a straight-in and overhead approach. These parameters all affect the entry autopilot, so they should be set before Entry Interface (EI).
Table \ref{tab:LandingSites} shows the list of landing sites currently supported by SSU.

\begin{table*}[tb]
  \centering
  \begin{tabular}{l | l l l}
    \textbf{SITE} & \textbf{Location} & \textbf{PRI RWY} & \textbf{SEC RWY} \\
    \hline
    1 & KSC & KSC 15 & KSC 33 \\
    3 & Moron & MRN 20 & MRN 02 \\
    13 & Moron & MRN 20 & MRN 02 \\
    20 & St. John's International & YYT 29 & YYT 11 \\
    23 & Lajes & LAJ 15 & LAJ 33 \\
    29 & Istres & FMI 33 & FMI 15 \\
    32 & Diego Garcia & JDG 31 & JDG 13 \\
    33 & RAAF Amberley/Tindall & AMB 15 & PTN 14 \\
    39 & Hao Atoll & HAO 12 & HAO 30 \\
    45 & Edwards AFB & EDW 22 & EDW 04
  \end{tabular}
  \caption{Landing Site Table}
  \label{tab:LandingSites}
\end{table*}
\end{multicols*}

\end{document}
